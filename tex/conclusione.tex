I risultati ottenuti per la frequenza di risonanza sono risultati essere generalmente concordi fra di loro, infatti gli scostamenti fra il valore atteso.... e i valori ottenuti sperimentalmente.... sono compresi tra il..... del valore stesso.
Si è ipotizzato che tale differenze siano dovute ai diversi fattori ambientali, prima fra tutti la  temperatura variabile delle componenti della schedadi acquisizione NI ELVIS II, dovuta all’effetto Joule.
È stata inoltre evidenziata una errata stima della resistenza totale ottenuta dai fit sulle ampiezze della tensione ai capi di ciascun componente e gli andamenti delle fasi sono risultati essere non del tutto consistenti con quelli attesi.
Le varie prove effettuate per cercare di risolvere il problema sono risultate vane, dal momento che non è stata ottenuta nessuna modifica rilevante sugli andamenti delle fasi, si è ipotizzato perciò che il problema sia legato al modo in cui vengono raccolti i dati dal software utilizzato per l’esperienza di laboratorio.
