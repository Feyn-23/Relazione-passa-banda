

In conclusione il comportamento del filtro è risultato essere in accordo con la teoria.
Per quanto riguarda il fenomeno della risonanza, dall'analisi delle ampiezze per $f_0$ sono stati
ottenuti i valori $f_0 = (18.95 \pm 0.05) kHz$ e $f_0 = (18.80 \pm 0.28)kHz$, mentre
da quella delle fasi $f_0 = (19.10 \pm 0.10)kHz$ e $f_0 = (19.35 \pm 0.15)kHz$ .
Si è ipotizzato che le discrepanze dal valore atteso  $f_0 = (19.13 \pm 0.14) kHz$, dove presenti, siano dovute ai diversi fattori ambientali, primo fra tutti la temperatura
variabile delle componenti della scheda di acquisizione NI ELVIS II, fenomeno dovuto all’effetto Joule.

È stata inoltre evidenziata un'errata stima iniziale della resistenza totale del circuito, in quanto i valori di tale
grandezza sono risultati sistematicamente superiori in tutti i fit realizzati.
Per avere un'idea di quanto tale valore risulti superiore a quello atteso, $R_{\text{tot}} = (1172.5 \pm 1.6) \ \Omega$, si riporta
$R =( 2.15 \pm 0.15 )10^3 \ \Omega$, dato ottenuto nel caso del fit, piuttosto significativo, dell'ampiezza di $V_R$.
