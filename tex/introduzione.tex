In un circuito $RLC$ in regime sinusoidale si assiste al fenomeno fisico della risonanza; in particolare, dati i valori
caratteristici di resistenza,induttanza e capacità del circuito, si osserva tale fenomeno in corrispondenza di un preciso
valore di frequenza, detta,appunto, di risonanza. La larghezza della curva di risonanza è legato al valore del cosiddetto
fattore di qualità $Q$ del circuito determinato dai componenti circuitali utilizzate.
È stato utilizzato un generatore di tensione sinusoidale il cui scopo principale è stato quello di indurre oscillazioni
della corrente all'interno circuito e per valutare la sua risposta in frequenza.

La frequenza di risonanza si ha quando la tensione generata ai capi del circuito oscilla con pulsazione
\begin{equation}
    \omega_0 = \frac{1}{\sqrt{L C}}
\end{equation}
In corrispondenza di questo valore il comportamento previsto è che la differenza di potenziale ai capi della
resistenza sia in fase con quella ai capi del generatore e che, inoltre, sia massimizzata l'ampiezza di tali segnali
di tensione.
Tale circuito si classifica tra i filtri di tipo "passa banda", ovvero tra quei dispositivi passivi che permettono il
passaggio di frequenze all'interno di un dato intervallo, la cosiddetta banda passante, ed attenua le frequenze al
di fuori di esso.
% #TODO valuta se mettere osservazione sulla corrente
%poichè in un intorno della frequenza di risonanza, la
%corrente che scorre è maggiore, per poi diminuire di ampiezza a mano mano che ci si allontana dalla condizione sopra citata.
%
