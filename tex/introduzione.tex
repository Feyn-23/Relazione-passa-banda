In un circuito $RLC$ in regime sinusoidale si assiste al fenomeno fisico della risonanza; in particolare, dati i valori
caratteristici di resistenza, induttanza e capacità del circuito, si osserva tale fenomeno in corrispondenza di un preciso
valore di frequenza, detta, appunto, di risonanza.
La larghezza della curva di risonanza è legata al valore del cosiddetto
fattore di qualità $Q$, determinato dai componenti circuitali utilizzati.
%È stato utilizzato un generatore di tensione sinusoidale il cui scopo principale è stato quello di indurre oscillazioni
%della corrente all'interno circuito e per valutarne la risposta in frequenza.

Quando la tensione nel circuito oscilla con pulsazione
\begin{equation}\label{eq:res-pulsation}
    \omega_0 = \frac{1}{\sqrt{L C}}
\end{equation}
si è in condizione di risonanza con frequenza $f_0$.
In corrispondenza di questo valore ci si aspetta che la differenza di potenziale ai capi della
resistenza sia in fase con quella ai capi del generatore e che, inoltre, l'ampiezza di tali segnali
di tensione, così come la corrente circolante, sia massimizzata.
Tale circuito si classifica tra i filtri di tipo ``passa banda", ovvero tra quei dispositivi passivi che permettono il
passaggio di frequenze all'interno di un dato intervallo, la cosiddetta banda passante, ed attenua le frequenze al
di fuori di esso.