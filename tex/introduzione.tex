In un circuito RLC in regime sinusoidale i valori di resistenza R, di capacità C e di induttanza L determinano la pulsazione di risonanza omega0 e il fattore di qualita Q.
Il circuito utilizzato per questa esperienza di laboratorio è costituito da un generatore sinusoidale il cui scopo principale è quello di indurre oscillazioni della corrente nel circuito; la frequenza del generatore, inoltre, influenza l’ampiezza e lo sfasamento delle tensione ai capi dei componenti scelti per l’esperienza di laboratorio.
La frequenza di risonanza si ha quando la tensione generata ai capi del circuito ha come pulsazione.....
In corrispondenza di questo valore il comportamento previsto è che la differenza di potenziale sulla resistenza sia in fase con quella del generatore e sia massimizzata, mentre ci si aspetta che  sugli altri due componenti il valore sia coincidente.
Un circuito di questo tipo viene anche chiamato “passa banda”, poichè in un intorno della frequenza di risonanza, la corrente che scorre è maggiore, per poi diminuire di ampiezza a mano mano che ci si allontana dalla condizione sopra citata.
