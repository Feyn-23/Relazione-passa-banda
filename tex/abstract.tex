In questa esperienza di laboratorio si è analizzato il comportamento di un circuito RLC serie, sottoposto ad una tensione
in regime sinusoidale con pulsazione variabile.
In particolar modo, si è cercato di stimare il valore della frequenza di risonanza, andando a confrontare il comportamento
del circuito per valori della frequenza prossimi
al valore cercato per poi, acquisendo e analizzando i dati relativi alla tensione in entrata e nei rami, verificare
l’andamento atteso dell’ampiezza della tensione.

Le stime più significative della frequenza di risonanza sono $f_0 = (18.95 \pm 0.05) kHz$, valore ottenuto dai parametri
del fit sulla curva dell’ampiezza della resistenza e $f_0 = (19.10 \pm 0.10)kHz$, valore per cui la tensione su di essa
ha fase nulla, compatibili con il valore atteso pari a $f_0 = (19.13 \pm 0.14) kHz$.


