In questa esperienza di laboratorio si è analizzato il comportamento di un circuito RLC serie sottoposto a una tensione
in regime sinusoidale con pulsazione variabile.
In particolar modo si è cercato di stimare il valore della frequenza di risonanza, andando a confrontare il comportamento
del circuito per valori della frequenza prossimi
al valore cercato per poi, acquisendo e analizzando i dati relativi alla tensione in entrata e nei rami, verificare
l’andamento atteso dell’ampiezza della tensione.

Uno dei valori della frequenza di risonanza ottenuto per via sperimentale mediante l'utilizzo dei parametri del fit dell'ampiezza
della tensione sulal resistenza risulta essere $f = (18.95 \pm 0.05) kHz$, compatible con il valore
della frequenza di risonanza atteso pari a $f = (19.13 \pm 0.14) kHz$. %% #TODO valuta se scrivere del valore della fase


