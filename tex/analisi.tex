Osservazione qualitativa degli andamenti (si puo cambiare)
Una volta scelti i valori dei componenti da utilizzare per l’esperienza di laboratorio e calcolato il valore atteso della frequenza di risonanza che, come citato precedentemente, corrisponde a...., è stato analizzato dal punto di vista qualitativo il comportamento del circuito per verificare che fosse in accordo con quanto previsto e descritto nella sezione “Introduzione”.
Come si può osservare nella figura 2., andando a studiare frequenze costanti della tensione in ingresso, prossime al valore di risonanza, l’andamento della tensione sulla resistenza è risultato essere in fase con quello indotto sul generatore, come ci si aspettava dal punto di vista teorico.
L’anomalia riscontrata sul primo periodo delle acquisizioni effettuate a tensioni costanti verrà analizzata e descritta nella sezione “Appendice” di seguito.

Analisi delle ampiezze
Gli andamenti della tensione ai capi di ciascun componente e del circuito stesso, ottenuti da calcoli svolti dal programma di acquisizione dati Labview al variare della frequenza, sono mostrati nella figura 3.
L’incertezza associata alle tensione è stata ottenuta come la deviazione standard delle misure associate all’ampiezza della tensione agli estremi, che sono perciò posti costanti, ottenendo quindi come valore …..
Qui di seguito sono mostrate le funzioni che descrivono l’andamento del modulo dell’ampiezza ai capi di ciascun componente e il procedimento adottato per ricarvarle viene analizzato con più attenzione nella sezione “Appendice”.










Rr rappresenta la resistenza escludendo però quelle interne a generatore e induttanza, mentre omega indica la pulsazione della tensione applicata.
Andando ad analizzare l’andamento della tensione ai capi di ciascun componente e del circuito stesso, si può notare che l’ampiezza della tensione indotta nel circuito prensenta un picco per valori della frequenza prossimi a quella di risonanza per poi presentare valori più bassi via via che ci si allontana da questa condizione.
Questo effetto è dovuto alla resistenza interna del generatore che provoca una caduta di potenziale proporzionale alla corrente che attraversa il circuito, il cui valore in modulo è massimo proprio in corrispondenza della frequenza di risonanza.
I valori del chi quadro che sono stati ottenuti dai fit sui tre componenti sono...



Scrivere come si è ottenuta la frequenza di risonanza dai grafici dei fit







Analisi delle fasi
Le funzioni che descrivono gli andamenti attesi delle fasi sono le seguenti:









Scrivere dei fit sui dati


