Al fine di studiare il comportamento del circuito analizzato in questa esperienza di laboratorio, si è utilizzato il concetto di impedenza, che generalizza il concetto più specifico di resistenza nel caso di correnti sinusoidali.
L’impedenza totale è data dalla somma delle impedenze di ciascun componente, sfruttando il fatto che questi siano posti in serie. Perciò avremo...


dove j indica l’unità immaginaria, mentre il termine tra parentesi indica la reattanza, costituita in parte dal contributo induttivo e in parte da quello capacitivo.
Per trovare la corrente nel circuito invece è stata applicata la legge di Ohm simbolica, sfruttando il formalismo dei fasori e, moltiplicando poi il valore ottenuto per l’impedenza ai campi di ciascun componente, si può trovare l’andamento della tensione su di essi.

Si e ipotizzato che la resistenza ag- giuntiva potesse essere dovuta a un difetto del capacitore, unico componente per
il quale non si era precedentemente cercato un valore di resistenza, non dovendola avere. Tuttavia, anche con una successiva
verifica, isolandolo dal circuito, esso risulta non essere il problema. Con prove ripetute si ottiene lo stesso risultato e si
`e ipotizzato quindi che la causa dell’anomalia possa essere il modo in cui i dati sono raccolti dal LabView.