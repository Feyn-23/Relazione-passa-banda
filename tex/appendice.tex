
\begin{itemize}
    \item I parametri del fit dell'ampiezza di $V_R$ riportati nell'analisi sono stati ottenuti considerando come parametro libero
    $R^*$, rappresentante la resistenza aggiuntiva a $R_r$ che si oppone al fluire della
    corrente, in luogo di $R_{tot}$.
    Questa modifica è stata fatta a posteriori, osservando che l'esito del fit mostrava una
    correlazione tra i parametri $R_r$ e $R_{\text{tot}}$ e quindi un'hessiana (ed una matrice di covarianza) non definita
    positiva.
    Il problema è stato risolto con la modifica suddetta, ottenendo anche un miglior valore del coefficiente
    di determinazione.
    \item Per quanto riguarda il valore di $R_{\text{tot}}$, superiore a quello atteso, si è dapprima ipotizzato
    che ciò fosse dovuto ad una mancata valutazione di un possibile effetto resistivo del condensatore.
    Tuttavia, anche isolandolo e cercando di valutarne la resistenza con il multimetro digitale, non è stata ottenuta
    nessuna conferma di questa ipotesi.
    Si suppone quindi che questa anomalia sia dovuta alla metodologia di acquisizione dati tramite \emph{LabView}.
    \item Analizzando il circuito con il formalismo dei fasori (Perfetti,[2]) e considerando il concetto di impedenza dei bipoli elementari, che
    generalizza il concetto di resistenza, si ha che l'impedenza totale corrisponde a ($j$ indica l'unità immaginaria):
    \[
        \mathbb{Z}_{tot} = \mathbb{Z}_R + \mathbb{Z}_L + \mathbb{Z}_C = R + j \left( \omega L - \frac{1}{\omega C} \right)
    \]
    essendo i componenti in serie. Applicando ora la legge di Ohm simbolica si ottiene la corrente che circola nel circuito,
    dalla quale si ottiene la tensione agli estremi di un certo componente moltiplicando per la relativa impedenza.
    A titolo di esempio, per il resistore si ottiene il numero complesso
    \[
        V_R = \frac{R V_0}{ R + j \left( \omega L - \frac{1}{\omega C} \right)}
    \]
    da cui si estraggono ampiezza (modulo del numero complesso) e fase, corrispondenti alle equazioni  (\ref{eq:amp-V_R})
    e (\ref{eq:fase_R}),
    rispettivamente.
\end{itemize}








