

\begin{figure}
    \centering
    \begin{circuitikz}
        % Circuit

        \draw[line width=0.7]
        (2,5) to [sinusoidal voltage source, l_=$V_{\text{gen}}$] (2,1)
        (2,5) to [resistor, l_=$R$]  ++(8,0)to [inductor, l_=$L$] ++(0,-4) to [capacitor, n=cap] +(-8,0)
        (cap.s) node[above]{$C$};

        %% capi di gen
        \draw[line width=0.7]
        (3,2) to[short,*-] (2,2);
        \draw[line width=0.7]
        (3,4) to[short,*-] (2,4);
        %% capi di R
        \draw[line width=0.7]
        (5,4) to[short,*-] (5,5);
        \draw[line width=0.7]
        (7,4) to[short,*-] (7,5);
        %% capi di L
        \draw[line width=0.7]
        (9,4) to[short,*-] (10,4);
        \draw[line width=0.7]
        (9,2) to[short,*-] (10,2);
        %% capi di C
        \draw[line width=0.7]
        (5,2) to[short,*-] (5,1);
        \draw[line width=0.7]
        (7,2) to[short,*-] (7,1);
%            \draw[line width=0.7]
%            (2,7) to [sinusoidal voltage source, l_=$V_S$] (2,1)
%            (2,7) to [resistor, l_=$R$]  ++(8,0)to [inductor, l_=$L$] ++(0,-6) to [capacitor,n=cap] +(-8,0) (cap.s) node[above]{$C$};
%            \draw[line width=0.7]
%            (3,3) to[short,*-] (2,3);
%            \draw[line width=0.7]
%            (3,5) to[short,*-] (2,5);
%            \draw[line width=0.7]
%            (5,6) to[short,*-] (5,7);
%            \draw[line width=0.7]
%            (7,6) to[short,*-] (7,7);
%            \draw[line width=0.7]
%            (9,5) to[short,*-] (10,5);
%            \draw[line width=0.7]
%            (9,3) to[short,*-] (10,3);
%            \draw[line width=0.7]
%            (5,2) to[short,*-] (5,1);
%            \draw[line width=0.7]
%            (7,2) to[short,*-] (7,1);
    \end{circuitikz}
    \caption{\emph{Schema del circuito realizzato.}}
    \label{fig:circuit}
\end{figure}
Il circuito realizzato, illustrato in Figura \ref{fig:circuit}, è composto da una resistore, un induttore e da un condensatore, collegati in serie
sulla breadbord della scheda di acquisizione dati NI Elvis II\textregistered.

I valori dei componenti utilizzati sono i seguenti: $R_r = (996.7 \pm 1.4) \ \Omega$, $C = (1.46 \pm 0.01) \times 10^{-9} \ F$,
$L = (47.41 \pm 0.05) \times 10^{-2} H$ con resistenza interna $R_{\text{ind}} = (125.82 \pm 0.16) \ \Omega$. La resistenza interna
del generatore ammonta a $R_{gen} = 50 \Omega$. Questi rappresentano il valor medio estratto da misurazioni ripetute.
Avendo utilizzato il multimetro digitale di ELVIS II, per effettuare le precedenti misurazioni sono state calcolate le
incertezze in accordo con quanto riportato sulla scheda tecnica di tale strumento. Si è valutata l'entità di possibili
fluttuazioni statistiche per poi confrontare i dati ottenuti con i precedenti.
Sono risultate essere predominanti le incertezze calcolate secondo le specifiche della scheda di acquisizione.
I valori precedenti sono stati scelti in maniera tale da ottenere
un fattore di qualità $Q$ ragionevole per una buona osservazione del fenomeno della risonanza.


