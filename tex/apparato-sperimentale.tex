



Il circuito utilizzato è composto da una resistore, un induttore e da un condensatore collegati in serie
sulla breadbord della scheda di acquisizione dati NI Elvis II®.
I valori scelti per i componenti sono i seguenti: $R_r = (996.7 \pm 1.4) \ \Omega$, $C = (1.46 \pm 0.01)10^{-9} \ F$,
$L = (47.41 \pm 0.05)10^{-2} H$ con resistenza interna $R_{\text{ind}} = (125.82 \pm 0.16) \ \Omega$. La resistenza interna
del generatore ammonta a $R_{gen} = 50 \Omega$.
Avendo utilizzato il multimetro digitale di ELVIS II per effettuare le precedenti misurazioni sono state calcolate le
incertezze in accordo con quanto riportato sulla scheda tecnica di tale strumento. Si è valutata l'entità delle possibili
fluttuazioni statistiche mediante misurazioni ripetute per poi confrontare i dati ottenuti con i precedenti. Sono risultate
essere predominanti le incertezze standard. %#TODO come cazzo le chiamo?


I valori precedenti sono stati scelti in maniera tale da ottenere
un fattore di qualità $Q$ ragionevole per una buona osservazione del fenomeno della risonanza. %#TODO commento su Q
%#TODO osservcazione sample rate elvis sulla scelta dei valori

I capi di ogni componente, cosi come gli estremi del circuito, illustrato in Figura \ref{fig:circuit}, sono stati collegati
ad un canale della scheda per poter eseguire la lettura dei valori di ampiezza e fase della tensione su di essi.
Le suddette misure sono state ottenute tramite l'utilizzo del subVI “Extract Single Tone Information” del software di acquisizione
\emph{LabView}.

Per realizzare questa esperienza di laboratorio sono stati raccolti $500$ campioni a frequenza costante in varie condizioni
con una frequenza di $250000$ campioni al secondo, poi, effettuando uno sweep di frequenza, si sono raccolti dati ad
intervalli di $50Hz$ fra $5kHz$ e $35KHz$. %#TODO non proprio cosi uguale

    \begin{figure}
        \centering
        \begin{circuitikz}
            % Circuit
%            \draw[line width=0.7]
%            (2,7) to [sinusoidal voltage source, l_=$V_S$] (2,1)
%            (2,7) to [resistor, l_=$R$]  ++(8,0)to [inductor, l_=$L$] ++(0,-6) to [capacitor,n=cap] +(-8,0) (cap.s) node[above]{$C$};
%            \draw[line width=0.7]
%            (3,3) to[short,*-] (2,3);
%            \draw[line width=0.7]
%            (3,5) to[short,*-] (2,5);
%            \draw[line width=0.7]
%            (5,6) to[short,*-] (5,7);
%            \draw[line width=0.7]
%            (7,6) to[short,*-] (7,7);
%            \draw[line width=0.7]
%            (9,5) to[short,*-] (10,5);
%            \draw[line width=0.7]
%            (9,3) to[short,*-] (10,3);
%            \draw[line width=0.7]
%            (5,2) to[short,*-] (5,1);
%            \draw[line width=0.7]
%            (7,2) to[short,*-] (7,1);
            \draw[line width=0.7]
            (2,5) to [sinusoidal voltage source, l_=$V_{\text{gen}}$] (2,1)
            (2,5) to [resistor, l_=$R$]  ++(8,0)to [inductor, l_=$L$] ++(0,-4) to [capacitor, n=cap] +(-8,0)
            (cap.s) node[above]{$C$};

            %% capi di gen
            \draw[line width=0.7]
            (3,2) to[short,*-] (2,2);
            \draw[line width=0.7]
            (3,4) to[short,*-] (2,4);
            %% capi di R
            \draw[line width=0.7]
            (5,4) to[short,*-] (5,5);
            \draw[line width=0.7]
            (7,4) to[short,*-] (7,5);
            %% capi di L
            \draw[line width=0.7]
            (9,4) to[short,*-] (10,4);
            \draw[line width=0.7]
            (9,2) to[short,*-] (10,2);
            %% capi di C
            \draw[line width=0.7]
            (5,2) to[short,*-] (5,1);
            \draw[line width=0.7]
            (7,2) to[short,*-] (7,1);

            % Grid
% 		\draw[help lines] (0,0) grid (10,10)	;
        \end{circuitikz}
        \caption{\emph{Schema del circuito realizzato.}}
        \label{fig:circuit}
    \end{figure}

